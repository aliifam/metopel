\section{Conclussion}

In conclusion, this journal highlights the transformative potential of Docker in accelerating development and deployment in the SDLC. The research findings indicate that Docker offers numerous advantages, including increased efficiency, improved collaboration, and streamlined deployment processes \citep{bisht2016containerization,arora2018performance}. However, challenges related to security and resource utilization should be addressed to ensure successful Docker adoption.

The findings of this study contribute to the growing body of knowledge on Docker's impact on the SDLC. Further research is encouraged to explore Docker's potential in specific domains and industries, considering different development methodologies and organizational contexts. As Docker continues to evolve, it promises to revolutionize software development practices and enable organizations to deliver high-quality applications more efficiently and effectively.