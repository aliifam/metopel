\section{Pengertian Paradigma Penelitian}
Paradigma penelitian mengacu pada kerangka berpikir dan perspektif filosofis yang menjadi landasan utama dalam melakukan penelitian \citep{denzin2011sage}. Paradigma ini mengatur cara berpikir, menyusun pertanyaan penelitian, merancang metode, dan menganalisis data. Dalam lingkup perguruan tinggi, terdapat beberapa paradigma penelitian utama yang umum digunakan, yaitu paradigma positivisme, paradigma konstruktivisme, dan paradigma kritis \citep{butsi2019memahami}.

\subsection{Paradigma Positivisme}
Paradigma ini berfokus pada penelitian yang bersifat objektif dan kuantitatif. Peneliti berusaha mencari fakta dan hukum yang berlaku umum dan dapat diukur secara empiris. Metode yang sering digunakan dalam paradigma ini adalah eksperimen, survei, dan analisis statistik. Tujuan utama dari paradigma ini adalah mencari jawaban yang obyektif dan dapat diandalkan.

\subsection{Paradigma Konstruktivisme}
Paradigma ini menekankan pada konstruksi sosial dari realitas dan memandang pengetahuan sebagai hasil dari interaksi sosial. Peneliti dalam paradigma ini lebih fokus pada pemahaman makna yang diberikan oleh individu atau kelompok terhadap suatu fenomena. Metode penelitian yang cocok untuk paradigma ini adalah wawancara, observasi partisipatif, dan analisis naratif.

\subsection{Paradigma Kritis}
Paradigma ini memandang penelitian sebagai upaya untuk mengkritisi dan membongkar struktur sosial, politik, dan ekonomi yang ada. Tujuan dari penelitian dalam paradigma ini adalah untuk menghasilkan perubahan sosial dan transformasi masyarakat. Metode penelitian yang relevan meliputi analisis teks kritis, penelitian aksi, dan pendekatan partisipatif.