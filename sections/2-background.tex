\section{Background Study}

The background study section provides an in-depth understanding of Docker containerization technology. It explains the fundamental concepts and principles underlying Docker, including containerization, images, and containers. The section highlights the advantages of Docker over traditional virtualization approaches, such as lightweight resource utilization and faster startup times. Furthermore, it explores the key features of Docker that make it an ideal tool for enhancing development and deployment in the SDLC. These features include container isolation, image versioning, and efficient resource management.

To support the benefits of Docker in the SDLC, the background study section also discusses relevant literature on Docker adoption in various industries and domains. It showcases real-world examples of organizations that have successfully leveraged Docker to improve their development and deployment processes. These case studies highlight the transformative potential of Docker in terms of increased efficiency, reduced overhead, and improved collaboration among development teams.

\begin{figure}[h]
    \centering
    \includegraphics[width=0.7\textwidth]{./resources/images/horizontal-logo-monochromatic-white.png}
    \caption{Docker Logo}
    \label{fig:docker-logo}
\end{figure}

\clearpage

In this article, we have looked at what containers are and how they are used in software development. We have also looked at the different tools and technologies that are used to create and manage containers, including Docker, Docker Compose, Docker Swarm, and Docker Machine. Finally, we have looked at some of the key features of Docker Swarm, including service discovery, load balancing, scaling, rolling updates, and security.

\begin{table}[h]
    \begin{tabular}{|l|l|}
    \hline
    \textbf{Virtual Machines}             & \textbf{Containers}                            \\ \hline
    Heavyweight.                          & Lightweight.                                   \\
    Limited performance.                  & Native performance.                            \\
    Each VM runs in its own OS.           & All containers share the host OS.              \\
    Startup time in minutes.              & Startup time in milliseconds.                  \\
    Allocates required memory.            & Requires less memory space.                    \\
    Fully isolated and hence more secure. & Process-level isolation, possibly less secure. \\ \hline
    \end{tabular}
\end{table}

To understand the fundamental concepts and advantages of Docker, it is essential to delve into the background study. Docker provides lightweight Linux containers that ensure consistent development and deployment environments \cite{merkel2014docker}. It offers reproducibility and simplifies the process of creating portable software packages \cite{boettiger2015introduction}. 