\section{Background Study}

\begin{figure}[h]
    \centering
    \includegraphics[width=0.7\textwidth]{./resources/images/horizontal-logo-monochromatic-white.png}
    \caption{Docker Logo}
    \label{fig:docker-logo}
\end{figure}

\clearpage

In this article, we have looked at what containers are and how they are used in software development. We have also looked at the different tools and technologies that are used to create and manage containers, including Docker, Docker Compose, Docker Swarm, and Docker Machine. Finally, we have looked at some of the key features of Docker Swarm, including service discovery, load balancing, scaling, rolling updates, and security.

\begin{table}[h]
    \begin{tabular}{|l|l|}
    \hline
    \textbf{Virtual Machines}             & \textbf{Containers}                            \\ \hline
    Heavyweight.                          & Lightweight.                                   \\
    Limited performance.                  & Native performance.                            \\
    Each VM runs in its own OS.           & All containers share the host OS.              \\
    Startup time in minutes.              & Startup time in milliseconds.                  \\
    Allocates required memory.            & Requires less memory space.                    \\
    Fully isolated and hence more secure. & Process-level isolation, possibly less secure. \\ \hline
    \end{tabular}
\end{table}