\section{Results}

The results demonstrate that Docker significantly accelerates development and deployment processes in the SDLC. It enables efficient collaboration, environment reproducibility, and code portability \citep{fink2018docker,franken2016using}. Docker also streamlines deployment by simplifying packaging and distribution, resulting in faster and more reliable deployments \citep{boettiger2017package,tanenbaum2019introduction}.

The results section presents the findings of the research, showcasing the benefits and challenges of adopting Docker in different stages of the SDLC. It demonstrates how Docker accelerates development processes by enabling efficient collaboration, facilitating environment reproducibility, and improving code portability. The section highlights the positive impact of Docker on testing strategies, emphasizing its ability to provide isolated and consistent environments for testing purposes. Furthermore, it explores how Docker streamlines deployment by simplifying the packaging and distribution of applications, leading to faster and more reliable deployments.

The results also address the challenges associated with Docker adoption, such as container security, resource overhead, and the learning curve for developers. The section discusses mitigation strategies and best practices to overcome these challenges, ensuring successful integration of Docker into the SDLC.