\section{Pendahuluan}

Perguruan tinggi sebagai lembaga pendidikan tinggi memiliki peran vital dalam pengembangan ilmu pengetahuan dan teknologi. Salah satu kegiatan inti yang menjadi fokus utama di perguruan tinggi adalah penelitian. Penelitian di perguruan tinggi tidak hanya menjadi sarana pengembangan ilmu, tetapi juga sebagai kontribusi untuk solusi permasalahan nyata dalam masyarakat. Oleh karena itu, paradigma penelitian yang digunakan oleh para peneliti dalam perguruan tinggi akan sangat mempengaruhi hasil dan dampak dari penelitian tersebut.

Beberapa paradigma penelitian yang umum digunakan dalam lingkup perguruan tinggi adalah paradigma positivisme, paradigma konstruktivisme, dan paradigma kritis \citep{creswell2014research, guba1994competing, smith1983quantitative}. Paradigma positivisme, seperti yang dijelaskan oleh Creswell \citep{creswell2014research}, berfokus pada penelitian yang bersifat objektif dan kuantitatif. Penelitian dalam paradigma ini bertujuan untuk mencari fakta dan hukum yang berlaku umum dan dapat diukur secara empiris. Sementara itu, paradigma konstruktivisme menekankan pada konstruksi sosial dari realitas dan makna yang diberikan oleh individu atau kelompok terhadap suatu fenomena \citep{guba1994competing}. Paradigma kritis, seperti yang diungkapkan oleh Smith \citep{smith1983quantitative}, memandang penelitian sebagai upaya untuk mengkritisi dan membongkar struktur sosial, politik, dan ekonomi yang ada.

Dalam implementasi paradigma penelitian di perguruan tinggi, faktor penting yang harus diperhatikan adalah kebebasan akademik, dukungan riset, dan kesadaran mengenai berbagai paradigma penelitian. Kebebasan akademik memungkinkan para peneliti untuk memilih paradigma penelitian yang sesuai dengan topik dan tujuan penelitian mereka. Perguruan tinggi juga harus menyediakan dukungan yang memadai dalam bentuk sumber daya, fasilitas, dan dana untuk mendukung penelitian. Selain itu, kesadaran dan pemahaman yang baik mengenai berbagai paradigma penelitian akan membuka peluang untuk mengadopsi pendekatan yang lebih beragam dan inovatif.
